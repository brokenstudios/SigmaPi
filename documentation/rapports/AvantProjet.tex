\documentclass[a4paper , 10pt]{article}
\usepackage[utf8]{inputenc}  
\usepackage[T1]{fontenc}
\usepackage{graphicx}
\graphicspath{{images/}} 

\usepackage{wrapfig} 
\usepackage{amsmath} 
\usepackage{gensymb}
\usepackage{fancyhdr}
\usepackage{lastpage}
\usepackage{textcomp}
\usepackage{color}
\usepackage{courier}
\usepackage{setspace}
\usepackage{listings}
\usepackage{ulem}
\usepackage{geometry}
\geometry{a4paper,
total = {140mm, 237mm},
left = 35mm,
top = 30mm}
\newcommand{\minus}{\scalebox{0.6}[1.0]{$-$}} 
\newcommand{\signature}[2]{%
 \par\nobreak\bigskip
 \begin{singlespace}%
 \mbox{}\hfill\begin{tabular}{p{8cm} }
     \rule{8cm}{0.5pt}\newline{}%
       \textbf{#1}\\%
      #2 %
 \end{tabular}%
 \end{singlespace}%
 \medskip%
}
\renewcommand{\lstlistingname}{Extrait de code}
\lstset{
language=Java,
basicstyle=\scriptsize\ttfamily, 
upquote=true,
aboveskip={1.5\baselineskip},
columns=fullflexible,
showstringspaces=false,
extendedchars=true,
frame=single,
breaklines=true,
showtabs=false,
showspaces=false,
showstringspaces=false,
identifierstyle=\ttfamily,
keywordstyle=\color[rgb]{0.2,0.2,0.2},
commentstyle=\color[rgb]{0.133,0.545,0.133},
stringstyle=\color[rgb]{0.627,0.126,0.941},
tabsize=5,
}
\pagestyle{fancy}
\fancyhead[R]{Projet final Sigmapi}
\fancyfoot[C]{\thepage}
\fancyhead[L]{Loïc Gillioz, Loïc Fracheboud}
\author{Loïc Gillioz, Loïc Fracheboud}
\title{Informatique - Projet final \\ \Huge Sigmapi (Patapon-like)}
\date{13 avril 2016}
\begin{document}
\maketitle
\begin{figure}[!h]
\centering
\includegraphics[scale=0.15]{images/icones}
\end{figure}
\pagebreak

\section{Description sommaire du jeu}
Sigmapi est basé sur le jeu sorti sur PSP "Patapon". Ce jeu connu un grand succès et c'est désormais une des références des jeux de rythme.
\paragraph*{Concept}
Le jeu place le joueur dans le rôle du dieu des Sigmapis. Les Sigmapis sont des indigènes dont la vocation principale est musicale.
Il suffit de quatre touches pour jouer.Chaque touche est assignée à un tambour particulier. La combinaison de plusieurs tambours permet de créer un rythme assigné à un ordre. Les ordres sont ainsi communiqués aux Sigmapis, selon un rythme en quatre temps. Quand un rythme est joué, les Sigmapis le répètent tout en exécutant l'action associée à ce rythme. Le joueur doit alors ruser pour utiliser les rythmes adéquats aux situations, pour terminer le niveau avec un minimum de pertes.
\paragraph*{Graphismes}
Le thème graphique sera revisité pour faire référence aux logos de la HES-SO Valais, et un style visuel propre à notre jeu sera développé. Le but sera de garder l'univers graphique originel (qui a été largement apprécié à la sortie du jeu original) et de le dériver à notre sauce.
\section{Parlons peu, parlons code}
Pour créer ce jeu, il nous faudra approcher les thèmes suivants :
\begin{itemize}
\item Gestion du temps (timers)
\item Gestion des animations avec des SpriteSheets
\item Utilisation de la physique pour les objets balistiques et des collisions
\item Utilisation des collisions pour appliquer les dégâts aux unités
\item Déplacement de a caméra selon la position des Sigmapis
\item Gestion des séquences de notes 
\item Création d'animations sans SpriteSheets (transformations par calcul)
\item Décors récursifs
\item Gestion des dégâts
\item Adaptation des aptitudes des Sigmapis selon leur niveau et race
\item Gestion des actions par fonction (p.ex. les archers tirent des flèches, lanciers des lances, les épéistes frappent mais ne lancent rien)
\end{itemize}


\vspace{30pt}
\begin{wrapfigure}[0]{r}{6cm}
\vspace{-52pt}
\centering
\includegraphics[scale=0.5]{signgillioz}
\end{wrapfigure}
\signature{Loïc Gillioz}{Sion, le 21 avril 2016} 

\begin{wrapfigure}[0]{r}{6cm}
\vspace{-52pt}
\centering
\includegraphics[scale=1]{signfracheboud}
\end{wrapfigure}
\signature{Loïc Fracheboud}{Sion, le 21 avril 2016} 
\end{document}

